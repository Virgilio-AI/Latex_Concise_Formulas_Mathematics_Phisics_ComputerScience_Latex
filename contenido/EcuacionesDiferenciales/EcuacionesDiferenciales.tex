\newpage
\begin{section}{Differential Ecuations}
	\begin{subsection}{linearity}
		$$ a_n(x) \frac{d^ny}{dx^n} + ... +  a_1(x) \frac{dy}{dx} + a_0(x)y = f(x)$$ 
	\end{subsection}
	\newpage
	\begin{subsection}{homogeneous ecuations}
		given:

		 $$M(x,y)dx + N(x,y)dy = 0$$
		 the ecuation is homogeneous if M and N are homogeneous functions of the same exponent
		cambio de variable $y=ux \; o \; x=uy \; , \; dy=xdu + udx $
		\autoref{subSec:homogeneous ecuation}{}
	\end{subsection}
	\begin{subsection}{homogeneous function of grade n}
		$$f(tx,ty) = t^nf(x,y)$$
		\label{subSec:homogeneous ecuation}
	\end{subsection}
	\begin{subsection}{Exact ED}
		para ser exacta tiene que cumplir dos condiciones 
	\begin{enumerate}
		\item $M(x,y)dx + N(x,y)dy = 0$
		\item $\frac{\partial M}{\partial y} = \frac{\partial N }{\partial x}$
	\end{enumerate}
	si no las cumple puedes usar el factor integrante para que cumpla
		\label{subSec:Exact ED}
		\autoref{subSec:integrant factor}{}


		para resolver toma en cuenta las siguientes dos cosas
		$$f(x,y) = \int M dx + g(y) = \int N dy + h(x) $$
	
	
		$$ \frac{ \partial F}{ \partial x} = M \; , \; \frac{ \partial F }{ \partial y} = N $$
	\end{subsection}
	\begin{subsection}{Bernouully ecuation}
	
		aplica cuando la ecuacion diferencial tiene la siguiente forma:
		
		$$P_0(x)\frac{dy}{dx} + P(x)y = F(x)y^n$$
		
		se hace el cambio de variable $u = y^{1-n}$ y se obtiene una ecuacion lineal
	
	
	\end{subsection}
	\begin{subsection}{Ricat Ecuation}
		 tiene la siguiente forma
		  $$y^{'} = Q(x)y^{2} + P(x)y + R(x)$$
		se hace la sustitucion $y=y_1 + u^{-1}$ 
	
	\end{subsection}

	\begin{subsection}{Cauchy Euler ecuation}
		se usa para resolver una ecuacion de segundo grado
	
	
		$$ax^2y'' + bxy' + cy = 0$$
		\begin{center}
		$y = x^r \;, \; x > 0 $
		\end{center}
	\end{subsection}
	
	\begin{subsection}{integrant factor}
	aplica cuando hay una $f(x,y) $ tal que $f(x,y)(ED) = exacta$
	
	\begin{itemize}
		\item si $ \frac{M_y - N_x}{N}$ es funcion solamente de x entonces $P(x) = \frac{M_y - N_x}{N}$
			\begin{center}
		$f(x) = e^{\int P(x)dx}$ es un factor de integracion
			\end{center}
	
		\item si $M_y - N_x = m\frac{N}{x} - n\frac{M}{y}$ entonces
			\begin{center}
				$f(x) = x^my^n$ es un factor de integracion
			\end{center}
	
	\end{itemize}
		\label{subSec:integrant factor}
		used by \autoref{subSec:Exact ED}{}
	
	\end{subsection}
	\begin{subsection}{Linear differential equations}
		$$\frac{dy}{dx} + P(x)y = q(x) $$
		$$u(x) = e^{\int P(x) dx }$$
		Sol =  $u(x)y = \int u(x)q(x)dx$
	
	\end{subsection}
	\begin{subsection}{Order Reduction}
		aplica cuando conoces una solucion de una ED Lineal homogenea de segundo orden
		
		$$y_2 = y_1 \int \frac{e^{- \int P(x) dx }}{y_1^{'}} dx $$
		
		$$y^{''} + P(x)y^{'} + q(x)y = 0$$
	\end{subsection}
	\newpage
	\begin{subsection}{Constant coeficients Ecuation}
		para poder resolver por este metodo tiene que ser una ecuacion lineal de coeficientes constantes
		de la forma $$y^{''}C_1 + y{'}C_2 + yC_3 = 0$$
		se hace la sustitucion $$y=e^{rx}$$ quedara una funcion cuadratica en terminos de r
		
		se puede llegar a usar la identidad de euler
		
		la solucion queda de la forma:
		$$y=C_1e^{r_1x} + C_2e^{r_2x}$$
		
		tambien puede servir:
		$$r = a + bi $$
		$$ y_1 = C_1*e^{ \alpha x } \cos (bx)$$
		$$ y_2 = C_2*e^{ \alpha x } \sen (bx) $$
		
		
		nota:
		si hay multiplicidad, ejemplo: $(r-1)^3 = 0 $
		$$y_h = e^{rx} + xe^{rx} + x^2e^{rx}$$
		siendo que r = 1 entonces:

		$$y_h = e^x + xe^x + x^2e^x$$	
	\end{subsection}
	\begin{subsection}{parameter variation}
		
		tienen la forma $k_1y^{''} + k_2y^{'} + k_3y = f(x) $
		
		$$u_1 = - \int \frac{y_2f(x)}{W} dx \;\; \;\;\; u_2 =  \int \frac{y_1f(x)}{W} dx $$
		
		$$
		W=
		\begin{vmatrix}
		y_1 & y_2 \\
			y_1^{'} & y_2^{'}
		\end{vmatrix}
		$$
		\begin{center}
			siendo$y_h$ la solucion de la ecuacion homogenea asociada
		
			$$ y_h = C_1y_1 + C_2y_2 $$
		
			y siendo $y_p$ la solucion definitiva
		
			$$ y_p = u_1y_1 + C_2y_2 $$
		
		\end{center}
	
	\end{subsection}
	\newpage
	\begin{subsection}{Indeterminate Coeficients}
		\begin{center}
			$r(x)$ = polinomio, exponencial, Seno, Coseno 
		\end{center}
		
		pasos:
		\begin{enumerate}
			\item Calcular $y_n$ es decir
				calcular la ecuacion homogenea relacionada, por coeficientes constantes
				
			\item Encontrat $y_p$
				\begin{enumerate}
					\item[caso 1] No hay funciones en comun con r(x)
		
						nota: tomar en cuenta el teorema de superposicion de soluciones
						si $$r(x) = x^3 + x +  10 \sen 8x$$
						simplemente se suman los proposiciones
						 $$y_p = Ax^3 + Bx^2 + Cx + D +  A \sen (8x) + B \cos (8x) $$
						 y lo mismo aplica para la multiplicacion
		
						\begin{enumerate}
							\item[-] $y^{''} + C_1y^{'} + c_2y = x^3 + x$
		
								proponer $\rightarrow y_p = Ax^3 + Bx^2 + Cx + D$
							\item[-] $y^{''} + C_1y^{'} + c_2y = 10 \sen 8x$
		
								proponer $\rightarrow y_p = A \sen (8x) + B \cos (8x) $
							\item[-] $y^{''} + C_1y^{'} + c_2y = 12 e^{5x} $
		
								proponer $\rightarrow y_p = Ae^{5x}$
						\end{enumerate}
					\item[caso 2] hay funciones que coinciden con r(x)
		
						simplemente multiplicar la funcion for x hasta que no hayas funciones en comun con x
						pero tiene que ser la $x^n$ mas pequena posible
				\end{enumerate}
		
		\end{enumerate}

	\end{subsection}

\end{section}
	




