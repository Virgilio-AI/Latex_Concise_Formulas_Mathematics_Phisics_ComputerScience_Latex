\newpage
\begin{section}{Arch Linux}
\begin{subsection}{Mantainance}
\begin{minted}{sh}
	#check file size
	du -sh .cache/
	#remove a file
	rm -rt .cache/
	#delete what you don't need in .config file
\end{minted}

specific mantainance:

\begin{minted}{sh}
	#check the failed systems
	systemctl --failed
	#check the systemd journal
	sudo journalctl -p 3-xb
	#if the system doesn't boots then ctrl+alt+shift then timeshift -restore
	#then update mirrors
	#clar chache

	#then to update the whole system use:
	sudo pacman -Syyu
	#to check system updates
	sudo pacman -Syu
	#if you wan't to remove all packages in the drive use
	sudo pacman -Scc
	#remove all unwanted dependencies
	paru -Yc 
	#remove orphan packages
	sudo pacman -Rns \$(pacman - Qdtq)
	#sudo pacman -Syyy Syncrhonise data use "mirror1"
\end{minted}

\end{subsection}
\begin{subsection}{Print in arch linux}

install packages: usbutils, lsusb, cups

use this to make cups usable
\begin{minted}{sh}
sudo systemct enable cups
sudo systemctl start cups
localhost:631

lp -d HP_Officejey_Pro_8600]
\end{minted}

\end{subsection}


\begin{subsection}{configure date and time}

\begin{minted}{sh}
hwclock --set --date = "04/32/2021 19:00:00"
hwclock -hctosys
\end{minted}

\end{subsection}

\begin{subsection}{Configure wireless}

\begin{minted}{sh}
	#when entering an iso
	iwctl
	#then in the ui

	#to list all available devices
	device list

	#to scan networks
	station <device> scan

	#to get newworks
	station <device> get-network

	#to connect to a network
	station <device> connect "<name of network>"

	#to check if the connection is staable
	ping -c s 8.8.8.8

	#don't forget before rebooting the iso run
	pacman nmtui
\end{minted}

 dwm basic configuration
 \begin{minted}{sh}
	 #MODKEY + shift + q to restart X server
	 startx # to start the X server
\end{minted}

\end{subsection}
\begin{subsection}{mount devices}
mount usb sticks:
\begin{minted}{sh}
	#to mount a usb stick
	mount /dev/sdb1 /mnt/<destination folder>
	#to unmount a sub stick
	umount /dev/sdb1
\end{minted}
mount an android device:
\begin{minted}{sh}
	#to mount and android device
	simple-mtpfs --device 1 tablet/

	#to unmount an android device
	fusermount -u /tablet

\end{minted}

\end{subsection}

\end{section}
